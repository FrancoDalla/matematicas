\documentclass[spanish,12pt,a4paper]{article}
\usepackage[spanish,es-lcroman]{babel}
\usepackage{mathtools}
\usepackage{enumitem}
\usepackage{amssymb}
\usepackage{amsthm}
\usepackage{centernot}

\newtheorem{theorem}{Theorem}
\newtheorem{lemma}[theorem]{Lema}
\renewcommand*{\proofname}{Demostración}
\renewcommand\qedsymbol{PQD}
%opening
\title{Análisis y Regresión Lineal – 2025}

\author{Lara Cellini, Bernabe Moro,\\Franco Dalla Gasperina, Joaquín Gabriel Sanchez}

\begin{document}
	
	\maketitle
	
	\section{Regresión Lineal Simple}
	
	\begin{enumerate}[label=\alph*)]
		\item Se define como la variable respuesta $y$, representante del rendimiento de kilos por hectárea de maní producido, el cual en nuestro conjunto de datos medidos posee un intervalo que va desde $539,43 \frac{kg}{hec}$ a $3234,21 \frac{kg}{hec}$. Además, se definen como variables independientes: la variable $x_1$ que representa el indice de tiempo, la cual va del año 1927 al año 2023; y la variable $x_2$ que representa la superficie sembrada en hectareas; y va de 45.606 a 452.118 hectáreas.
		
		El razonamiento detrás de la elección de la variable independiente ``Indice de tiempo'' es el de visualizar la relación entre los avances tecnológicos de producción en el rendimiento de las cosechas. Por otro lado, el motivo por el cual se puede elegir la ``Superficie sembrada en hectáreas'' es que tiene una mayor correlación con la variable dependiente ya que esta es uno de los factores que la componen; además, es notable que la relación entre la superficie sembrada y el rendimiento de producción no forman una relación 1 a 1, esto a juzgar por como el valor máximo de hectáreas sembradas es diez veces mayor a el mínimo observado, mientras que el rendimiento máximo medido es 6 veces mayor que el mínimo.
		
		Finalmente, la variable independiente fue seleccionada para este estudio debido a que nos interesa tener la capacidad de predecir el rendimiento de las cosechas anuales en base a las tecnologías utilizadas y/o la disponibilidad de tierras.
		
		\item ~
		\begin{table}[h!]
			\centering
			\bgroup
			\def\arraystretch{1.5}
			\begin{tabular}{|c|c|c|c|c|c|c|c|c|}\hline
				$Y$ & $\hat{y_i} = \hat{\beta_1}x_i+\hat{\beta_0}$ & $\hat{\sigma}^2$ & $R^2$ & $r$ & $IC(\beta_1)$ & $IC(\beta_0)$ & $ICM(Y)$ & $IP(y)$\\\hline
			\end{tabular}
			\egroup
		\end{table}
		\item
	\end{enumerate}
	
	\section{Regresión Lineal Múltiple}
	
	\begin{enumerate}[label=\alph*)]
		\item 
		\item 
		\item
	\end{enumerate}
\end{document}
